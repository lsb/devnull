\chapter{Parallel Schedule Synthesis}
\label{chap:4}


Programmers struggle to map applications into parallel algorithms. Going beyond the automatic schedule verification of the last chapter, we now examine how to automatically generate a schedule.  Consider two of the decisions that a programmer faces in manually designing a schedule:
\begin{itemize}
\item \textbf{Scheduling a single traversal.} Many computations contain sequential dependencies between nodes. One correct traversal over the full tree might then be sequential. However, if the sequential dependencies can be isolated to a subtree, an overall parallel traversal would be possible if it invokes a sequential traversal for just the isolated subtree. Whether such isolation is always possible is not obvious.
\item \textbf{Scheduling multiple traversals.} Programs such as browsers perform many traversals. Traversals might run one after another, concurrently, or be fused into one. These choices optimize for different aspects of the computation. Running two traversals in parallel improves scaling, but fusing them into one parallel traversal avoids overheads: the choice may depend on both the hardware and tree size. Which traversal sequence to use is not obvious.
\end{itemize}
These decisions explode the space of schedules. Today, programmers manually navigate the space by selecting a parallel schedule, judging its correctness,  and comparing its efficiency to alternative schedules. The tasks are expensive: programmers  globally reason about dependencies, develop prototypes for profiling, and whenever the functional specification changes, restart the process. 

This chapter explores the design and implementation of an attribute grammar that supports automatic schedule synthesis.
We examine several questions: 
\begin{itemize}
\item What programming constructs are enabled by schedule synthesis?
\item What is an algorithm to \emph{quickly} find a \emph{correct} schedule?
\item If multiple schedules are possible, how do we find a \emph{fast} one?
\end{itemize}
The following sections explore each question in turn. 



\section{Computer-Aided Programming with Schedule Sketching}
\label{sec:holes}
Automatic parallel schedule synthesis enables rich forms of parallel programming. The utility of these constructs is not obvious. An automation tool will automatically find a parallel schedule, so a natural conclusion would be to assume the programming interface simply hide all parallelization concerns and rely upon automatic parallelization internally. We found this to be largely true for writing small amounts of declarative data visualization code. However, in parallelizing the larger and more complicated CSS layout language, we encountered cases where the visualization designer needed to guide (or be guided by) the automation procedure. Likewise, we encountered the need for one programmer to communicate parallel structure to another. Automatic parallelization is insufficient in that it hides all parallelization details and controls, yet manual scheduling was too low-level and brittle.

Our solution is to provide a \emph{sketching} construct for specifying constraints on the schedule that the automatic parallelization algorithm must respect. The programmer chooses which parts of a schedule to write and relies upon the synthesizer to fill in the rest. We routinely sketched schedules in order to \emph{override schedule selection}, \emph{test} and \emph{debug} parallelization ideas, and \emph{enforceably communicate parallelization decisions} when sharing code with others. Discussed later in this chapter~\ref{??}, we also used the sketching mechanism to speed up automatic parallelization over specifications with many attributes or schedule patterns that challenge static analysis.

We revisit the specification of \hlang{} to demonstrate the sketching construct and its use for the above scenarios. First, depending on the expected memory size of target hardware, the programmer may choose a longer schedule with a smaller set of attributes computed in each one. Compare the three following schedule sketches:

\begin{align*}
%\begin{lstlisting}[mathescape,morekeywords={parPre,parPost}]
\hole_1 \\
\text{\textbf{parPost}} ~ \hole_2 ~ ; ~ \text{\textbf{parPre}} ~ \hole_3 \\
(\text{\textbf{parPost}} ~ \hole_4 ~ ; ~  \text{\textbf{parPost}}~ \hole_5) ~ ; ~\hole_6 
\end{align*}

The first specification leaves a \emph{hole} for the entire schedule. The synthesizer fills in every hole with a valid schedule term so that the resulting schedule is correct.  The entire first schedule is left as a hole, which is equivalent to requesting fully automatic parallelization. The second specification hardcodes the traversals but leaves holes for the attributes to schedule for each traversal. The final schedule sketch splits the \code{parPost} traversal in two in order to decrease the memory consumption in the first traversal. Like the second sketch, it does not specify the attributes, and like the first sketch, it does not specify the sequence of traversals to place at the end of the schedule. 

The ability to run a sketch through the synthesizer enables several forms of parallel program debugging. First, the synthesizer rejects programs that it cannot parallelize, so sketches can test programmer intuitions. For example, it could test the validity of the above idea of splitting apart the first \code{parPost} traversal. We could more explicitly test the underlying insight that the \code{w} and \code{h} attributes are separable:

\begin{align*}
& (\text{\textbf{parPost}} ~ \{ w  \} ~ ; ~  \text{\textbf{parPost}}~ \{ h \}) ~ ;  ~\hole_6
\end{align*}

The synthesizer can fill in $\hole_6$ to find a complete schedule. It provides an error if it cannot: the longest schedule prefix of traversals it could schedule. For the above example, the error distinguishes two possible mistakes. First, it fails with a prefix containing  \code{parPost \{ w \} ; parPost \{ h \}}, the first traversal can be split but the rest of the schedule has an unsatisfiable dependency. Otherwise, the prefix is empty and the traversals could not be split. We found the ability to test scheduling ideas to be particularly useful, for example, in determining partitions for nested text.

We provide another mechanism for debugging. The programmer may ask the synthesizer to \emph{enumerate} all valid solutions for a schedule sketch. The previous examples restricted themselves to only asking for one completion. However, for \hlang{}, the space of valid schedules is small enough that programmer could manually page through all possibles ones.


As our attribute grammars grew, we wrote sketches to help share code between programmers. Consider a program with a sketch such as the above. Upon receiving a grammar with it, a programmer knows the desired parallelization scheme. Furthermore, the synthesizer checks that edits to the functional specification do not violate the schedule. For example, the synthesizer would detect the addition of a feature that requires the addition of an extra traversal or serialization of a parallel one. We typically ignored changes that do not impact parallelization and applied more careful reasoning whenever a sketch was violated.  In this way, the ability to communicate and enforce schedule specifications helps separate concerns between defining layout feature logic and optimizing layout scheduling.


\section{Generalizing Holes to Syntactic Unification}
We provide a more expressive variant of holes for cases that require more high-level control than they provide. For example, we may want to specify that both the width and height are computed in the first traversal over a tree. The programmer should not have to specify the relative order of attributes for every type of node that computes them. Instead, we generalize the sketching construct to syntactic unification over scheduling terms. 

Programmers may specify constraints over schedule terms. For example, the following specification declares that the width and height attributes are computed in the first traversal of a sequence but do not specify their relative order:
\begin{align*}
member(&\text{w}, \hole_2), member(\text{w}, \hole_3),\\
member(&\text{h}, \hole_2), member(\text{h}, \hole_3),\\
\textbf{\emph{Sched}}  = ~[&~[ \hole_1, ~ [ [\text{HBox}, \hole_2], [\text{VBox}, \hole_3]]], ~\\
& ~ \textbf{seq},\\
& ~  [\text{\textbf{parPost}}, ~\hole_4]~]
\end{align*}
Term $\hole_1$ will unify with a traversal type and $\hole_2$ and $\hole_3$ will unify with a sequence of attributes that includes \code{w} and \code{h}. Finally, $\hole_4$ will unify with another sequence of terms where each specifies a node type and the sequence of attributes to schedule for it. Note the change in syntax.

Our scheduling language is an embedded domain specific language (EDSL~\cite{??}) in Prolog. The language of constraints is arbitrary Prolog. Thus, in the above example, \code{Sched} is a named Prolog variable that must be unified with the schedule constraints and the attribute grammar's functional dependencies. Likewise, unnamed variables $\hole_1$, $\hole_2$, and $\hole_3$ must unify with a correct schedule. Our system provides a library of traversal types such as \code{parPost} and combinators such as \code{seq}, and the attribute grammar introduces attribute terms such as \code{w} and \code{h}. The programmer then uses built-in Prolog predicates to constrain the result such as \code{member} for list membership. Likewise, they may use Prolog's ``,'' operator for conjunction and ``;'' for disjunction. 

We made several notable uses of the extended sketching constructs:
\begin{itemize}
\item \textbf{Attribute sets.} As in the above example, we specify \emph{unordered sets} of attributes for a traversal rather than an \emph{ordered sequence}. The synthesizer determines the order and any additional attributes. Likewise, different classes implementing the same interface share attributes of the same name, we wrote helper functions for expanding an interface attribute to schedule into the instances for the corresponding classes.
\item \textbf{Requiring parallelization.} We may specify that a traversal type unifies with a parallel form:
\begin{align*}
& (\hole_1 = \text{\textbf{parPost}} ; \hole_1 =  \text{\textbf{parPre}}), \\
& \textbf{\emph{Sched}} = [~[\hole_1, ~ \hole_2], ~ \text{\textbf{seq}}, ~ [\text{\textbf{parPost}}, ~\hole_3]]
\end{align*}
The sketch specifies a sequence of two traversals where the first traversal type ($\hole_1$) unifies with either  \code{parPost} or \code{parPre} traversal. The schedule does not specify what attributes are computed within the first traversal ($\hole_2$). Furthermore, instead of manually specifying the choice for every pass, we wrote a function that does so automatically.
\item \textbf{Nesting.} We use predicates to test the validity of partitioning into nested traversals. The challenge is to minimize the number of traversals put into a sequential partition. For example, if we thought a node belongs in a parallel partition, we would include that in the sketch;
\begin{align*}
member(&[\text{HBox}, \hole_1], TopDownVisits),\\
\textbf{\emph{Sched}}  =& [ ~ [\text{\textbf{nested}},  [\text{\textbf{parPre}}, TopDownVisits], ~\hole_2] ~ | ~ \hole_3]
\end{align*}
The schedule specifies that the first traversal is nested within an overall parallel preorder structure. The preorder portion must handle \code{HBox} nodes and may include other as well. The other partitions are defined by $\hole_2$, which has no additional constraints. The specification leaves remaining traversals unconstrained by $\hole_3$.
\item \textbf{Schedule heuristics.} For a simple optimization heuristic, we would bias parallel scheduling to use an alternating sequence of \code{parPre} ancd \code{parPost} traversal. The intuition is that long-running dependencies can often be satisfied under these two traversals so that the shortest schedule would be such an alternation. Less obviously, some dependencies require repetition of the same traversal pattern, so the heuristic is to bias to use of the alternate of whatever worked for the previous traversal and otherwise prioritize any other parallel traversal.
\end{itemize}

In all of the above cases, reasoning is in terms of the syntactic form. For example, the alternating traversal heuristic biases towards one traversal based on equality with the syntactic value of the the previous one. Richer forms of unification that extend beyond syntax may be applicable. As is, syntactic unification for guiding schedules already supports several key tasks.




\section{Fast Algorithm for Schedule Synthesis}
Our synthesizer takes an attribute grammar and a sketch as input, and outputs a set of schedules. We designed it to support multiple traversal types, multiple solutions, and rich attribute grammar and schedule sketching languages. Our initial implementation used the dependency analysis of \citeauthor{oag}~\cite{oag}, but it was too inflexible. Our new  algorithm optimizes for modularity and speed by using the following design:

\paragraph*{Simple enumerate-and-check} The algorithm enumerates schedules and checks which are correct. Checkers examine the use of individual traversal types and traversal compositors, and we wrote them to function independently of one another. Enumeration is simply syntactic. Combined, adding a new traversal type involves writing a checker and binding it to the proper place.

\paragraph*{Optimization}
Na\"{\i}ve enumerate-and-check is too slow. Without significantly changing the interface for adding checkers, we optimize synthesizing one schedule to be $O(n^3)$ in the number of attributes. Some features are still slow, such as nested traversals, so we introduce the optimizations of incrementalization, greediness, and greedy sketch unification. 

%We split the problem into 1) finding a correct partitioning of attributes for a fixed schedule of traversals and 2) finding correct schedules of traversals.

\subsection{The Algorithm}
\label{sec:sub:greedy}

We first discuss optimizations for  finding one correct schedule before considering finding many. Figure~\ref{fig:searchtrace} demonstrates an algorithm trace for enumerating schedules of \mbox{\hlang{}.} Figure~\ref{fig:synthalgs} shows the full algorithm.

\begin{figure}
\centering
\begin{minipage}{0.8\columnwidth}
\begin{lstlisting}[mathescape,escapechar=\`,morekeywords={parPre,parPost,;,||}]
parPre{x,y,w,h} `\BComment{incorrect: unsat \{x,w,h\}}`
parPre{y} `\Suppressnumber` `\Comment{correct: continue}`
$\ldots$ `/* expand subtree to schedule x, w, h */` $\ldots$ `\Reactivatenumber`
parPost{x,y,w,h} `\BComment{incorrect: unsat \{x,y\}}`
parPost{w,h} `\Comment{correct: continue}`
  _ ; parPre{x,y} `\BComment{correct: complete}`
  _ ; parPost{x,y} `\BComment{incorrect: unsat \{x,y\}}`
  _ ; (parPre{x} || _) `\Comment{correct: continue}`
    _ ; (_ || parPre{y}) `\BComment{correct: complete}`
    _ ; (_ || parPost{y}) `\BComment{incorrect: unsat \{y\}}`
  _ ; (parPre{y} || _) `\Comment{correct: continue}`
    _ ; (_ || parPre{x}) `\BComment{correct: complete}`
    _ ; (_ || parPost{x}) `\BComment{incorrect: unsat \{x\}}`
  _ ; (parPost{y} || _) `\BComment{incorrect: unsat \{y\}}`
  _ || parPre{x,y} `\BComment{incorrect: unsat \{x\}}`
  _ || (parPre{y} ; _) `\Comment{correct: continue}`
    _ || (_ ; parPre{x}) `\BComment{incorrect: unsat \{x\}}`
    _ || (_ ; parPost{x}) `\Suppressnumber` `\BComment{incorrect: unsat \{x\}}`
... `\Reactivatenumber`
parPost{w}  `\Comment{correct: continue}` 
  _ || parPre{x,y,h} `\Suppressnumber` `\BComment{incorrect: unsat \{x,h\}}`
$\ldots$ `\Reactivatenumber`
\end{lstlisting}
\end{minipage}
\caption{\textbf{Trace of synthesizing schedules for \hlang{}}. Note that scheduling of ``$||$'' does not use the optional greedy heuristic.}
\label{fig:searchtrace}
\end{figure}

Synthesizing one schedule is $O(A^3)$ in the number of attributes. The algorithm finds an increasingly long and correct prefix of the schedule (\emph{prefix expansion}). At each step, it tries different suffixes until one succeeds, where a suffix such as ``\sched{parPre\{x,y\}}'' is a traversal type and attributes to compute in it. When a correct suffix is found, it is appended to the prefix and the loop continues on to the next suffix. Finding one suffix involves trying different traversal types, and for each one, different attributes. Only the suffix needs to be checked (\emph{incremental checking}), and checking a suffix is fast (\emph{topological sort}). Finally, finding a set of attributes computable by a particular traversal type only requires $O(A)$ attempts (\emph{iterative refinement}).

We consider each optimization in turn:

\begin{enumerate}

\item \textbf{Prefix expansion.} The synthesizer searches for an increasingly large \emph{correct} schedule prefix. Every line of the trace represents a prefix. If a prefix is incorrect, no suffix will yield a correct schedule. Therefore, the only prefixes that get expanded are those that succeed (lines 2, 4, 7, 10, 15, 18). 

To synthesize only one schedule, only one increasingly large prefix is expanded. Line 2 has a correct prefix, so only ``\sched{parPre\{y\}}'' would be explored. Either no schedule is possible at all, or if there are any, one is guaranteed to exist in the expansion. In this case,  ``\sched{parPre\{y\} ; parPost\{w,h\} ; parPre\{x\}}'' would be found.

%Prefixes are trees, such as ``\sched{(p || q) ; (r || \_)}'', not just strings. 

\item \textbf{Incremental checking.} Line 4  checks prefix ``\sched{parPost\{w,h\}}'' for attributes ``w'' and ``h.'' Therefore, lines 5-17 can check the suffix added at each line without rechecking  ``\sched{parPost\{w,h\}}''. 

\item \textbf{Topological sort.} We optimize checking a suffix by topologically sorting the dependency graph of its attributes  (rule $\rrule{check_\beta}$ in the next subsection). Topologically sorting a graph is $O(V + E)$.  It is $O(A)$ in this case because $V = A$, and as the arity of semantic functions is generally small, $E$ is $O(A)$.  

\item \textbf{Iterative refinement.} The algorithm iteratively refines an over-approximation of what attributes can be computed in a suffix by removing under-approximations of what cannot. For example, the check in line 1 for \sched{parPre\{x,y,w,h\}} fails with error \lstinline|{x,w,h}|, which details the attributes with unsatisfiable dependencies. Computing fewer attributes cannot satisfy more dependencies, so no  subset of \lstinline|{x,w,h}| has satisfiable dependencies either. Therefore, the next check is on a set without them: \lstinline|{y}|. 

Subtraction of attributes repeats at most $A$ times before finding a solution or terminating on the empty set. Checking one refinement invokes the $O(A)$ topological sort. Put together, finding the attributes computable by a suffix is $O(A^2)$.
\end{enumerate}

\noindent Every traversal computes at least one attribute, so there are at most $A$ traversals. A constant number of traversal types are examined for each suffix, and synthesizing each one is $O(A^2)$. Synthesizing one schedule is therefore $O(A^3)$.  The \emph{greedy sketch unification} optimization from the next section may further optimize the synthesis of a single schedule.


\section{Schedule Enumeration}
We provide and optimize the ability to examine many schedules. Our approach benefits several scenarios: picking a fast schedule when many are possible, supporting scheduling language extensions that otherwise resist fast synthesis, and improving synthesis time when partial schedule knowledge is known.

We consider each scenario in turn:

\paragraph{}
\textbf{Autotuning} There may be an exponential number of schedules, and the choice of the fast is non-obvious. For example, shorter schedules incur less traversal overhead, but also generally expose less parallelism. Likewise, a short sequence of parallel traversals may behave worse than a long sequence when performed on hardware with limited memory. By enumerating all schedules, we can perform \emph{autotuning}: run performance tests to pick the best schedule for a particular architecture. There may be an exponential number of schedules, so we must somehow optimize the enumeration of those to test.

\paragraph{}
 \textbf{Scheduling extensions.} We provide optional scheduling language extensions, and fast synthesis in their presence requires optimization. For example, nested traversals require partitioning the set of nodes into distinct regions, but many partitions are possible. Partitioning does not enjoy the monotonicity property that we previously exploited and thus, on its own, is slow.

\paragraph{}
 \textbf{Faster synthesis.} Verification is linear-time in theory yet running the synthesizer to perform it, as is, takes cubic time.  If the programmer provides knowledge of the schedule, such as when recompiling the grammar, synthesis should execute faster. In the limit, providing full schedule knowledge should reduce synthesis time to that of verification.

\paragraph{}

We introduce several optimizations that, together, address the above scenarios. They optimize for when multiple schedules may be valid schedules, and except for backtracking, may also improve the process of finding one schedule. 
\begin{itemize}

\item \textbf{Backtracking.} To emit multiple schedules, we extend prefix expansion to also perform backtracking. After a schedule is fully completed or a suffix fails, the synthesizer backtracks to the most recent correct prefix. For example, line 8 is a complete and correct schedule. Backtracking returns to the earlier correct prefix of line 7 and tries the alternative suffix of line 9. 

%When synthesizing only one schedule, backtracking must be restricted to trying alternate suffixes until a correct one is found. Consider a grammar that cannot be scheduled and a search that reached a correct prefix that cannot be further expanded. The grammar should be rejected at this point. Otherwise, backtracking to an earlier prefix would make finding the error exponential in the number of traversals. 


\item \textbf{Greedy sketch unification.} We use sketches to prune the search. For example,  sketch

$$\text{\textbf{parPost}}~ \hole_1 ~ || ~\hole_2$$ 

enables skipping lines 1-3 because they do not start with a \sched{parPost} traversal. Lines 5-13 could also be skipped because the compositor is not ``$||$''. 

A sketch that provides a full schedule reduces synthesis to verification, which is $O(A)$ (topological sort). Sketching also enable features that otherwise require exponential search to still synthesize in $O(A^3)$. For example, scheduling nested regions is exponential in the number productions, but if just the production partitioning is sketched, synthesis for the remaining schedule terms is still only $O(A^3)$.

\item \textbf{Greedy attribute heuristic.} For any schedule ``\sched{p ; q}'', solving fewer attributes in $p$ will not enable solving $q$ with fewer traversals. Thus, to minimize the number of traversals, all such subsets are pruned. For example, as line 4 found \sched{parPost\{w,h\}}, line 19 skips ``\lstinline[morekeywords={parPost}]|parPost{w} ; _ |'' and proceeds to ``\sched{parPost\{w\} || _}''. 

Greediness reduces enumerating all schedules to only being exponential in the number of traversals. This is significant because, for example, our schedule for CSS has only 9 traversals.
\end{itemize}

In summary, synthesizing one schedule in our base language is $O(A^3)$, but emitting all of them is exponential. Likewise, scheduling language extensions such as nested traversals still support fast synthesis of surrounding terms when guided by sketches. Our optimizations optimize the process, such as by reducing synthesis complexity to that of verification when increasingly detailed sketches are provided.







\newsavebox{\goodsynth}
\begin{lrbox}{\goodsynth}% Store first listing
\begin{minipage}{1\columnwidth}
\begin{lstlisting}[mathescape,morekeywords={choose,while,yield, backtrack,unify,def},escapechar=+]
def synthFast(sketch):
  yield synth($\emptyset$, Attributes, sketch)

def synth(prev,rest,sketch):
  choose  $\otimes \in$ { +``;''+, +``$||$''+ }
  if $\otimes$ = +``;''+:
    choose $\alpha$ $\in$ { +``+parPre+''+, +``+parPost+''+, $\ldots$ }
    $A$ := iterativeRefine($\alpha$, prev, rest)
    if $A$ = rest:
      unify(sketch, $\alpha~A$)
      yield $\alpha~A$
    else if $A$ = $\emptyset$:    
      backtrack
    else:
      unify(sketch, $\alpha~A$ ; $rhs_1$)
      yield $\alpha~A$ ; synth(prev $\cup~A$, rest$~-~A$, $rhs_1$)
  else:
    unify(sketch, $lhs_2$ || $rhs_2$)
    choose $A \subset $ rest
    $p$ := synth(prev, $A$, $lhs_2$)
    $q$ := synth(prev, rest - $A$, $rhs_2$)
    yield $p$ || $q$

def iterativeRefine($\alpha$, prev, rest):
  overapproxA = rest
  do:
    X = check$_\alpha$(prev, overapproxA)
    overapproxA = overapproxA $-$ X
  while X $\neq \emptyset$
  yield overapproxA
  if nonGreedy:
    choose overapproxA$'$ $\subset$ overapproxA
    yield iterativeRefine($\alpha$, prev, overapproxA$'$)
\end{lstlisting}
\end{minipage}
\end{lrbox}




\begin{figure}
%\subfloat[\textbf{Na\"{\i}ve synthesis algorithm.} Phrased with search operators \texttt{choose,backtrack}, and multi-return \texttt{yield}. Function \texttt{unify} is ``='' in Prolog.]{\label{fig:badsynthalg} \usebox{\badsynth} } \\
%\subfloat[]{\label{fig:goodsynthalg} } \\
\usebox{\goodsynth}
\caption{\textbf{Optimized synthesis algorithm.} Lines 10,15,18: early unification with sketches. Lines 8,27: incremental checking. Line 26: iterative  refinement. Line 31: toggle minimal length schedules. Lines 12,28: pruning of traversals with unsatisfiable dependencies. }
\label{fig:synthalgs}
\end{figure}


\section{Evaluation}
We evaluate the automation capabilities of our schedule synthesizer for our case studies of data visualization and document layout. First, we examine whether the synthesizer could find parallelism and how much guidance it needed. Second, we evaluate whether our synthesis algorithm can achieve interactive or same-day compile times. Finally, we examine the quality of schedules: we measure the benefit of autotuning and the cost of our greedy heuristic.

\begin{figure}
\centering
\begin{tabular}{|l|r|r|r|r|r|r|} \hline
name & loc & 1st & sketch & found & avg \\ \hline
hbox++ & 305 & 5.6s & 9.6s & 54 & 2.7s   \\
spiral & 144 & 0.7s & 0.9s & 12 & 0.4s   \\
votes & 327 &  15.4s & 22.0s & 36 & 8.0s  \\
css & 1132 & 1919.6s & 65.1s & 100 & 445.4s \\ \hline
\end{tabular}
\caption{\textbf{Synthesizer speed:} \texttt{1st} is the time to first schedule without using a sketch. \texttt{sketch} is the time to first schedule using a sketch of the traversal sequence. \texttt{found} is the number of schedules found. \texttt{avg} is the average time to find a sketch.}
\label{fig:synthperf}
\end{figure}

\subsection{Automatic Parallelization}
We first evaluate whether the synthesizer automatically detected parallelism and the amount of schedule guidance we provided.

For all of the data visualizations (tree map, single and multiple time series, hbox, and sunburst), we successfully relied upon the synthesizer to automatically find parallelism. We performed an iterative design process where we would iterate between adding a feature and checking that the compiler could automatically parallelize it. Once the compiler accepted one functional specification, we would extend the specification with the next feature. When satisfied with the visualization, we would specify the sketch of parallel traversals, but only for communicating requirements to future programmers.

The CSS specification required guidance. On its own, the synthesizer would find a sequence of parallel preorder and postorder traversals. The exception is one traversal that requires nested partitions  for parallelization, so to improve synthesis times, we specified the structure of that traversal. Furthermore, due to the many cross-cutting data dependencies in CSS, we specified schedule sketches throughout the design process. The sketches ensured that extensions to the functional specification did not violate our understanding of the parallel behavior.

Figure~\ref{fig:synthperf} shows the lines of declarative code for each specification. The generated code was over a magnitude more depending on the compiler backend. The number of parallel traversals ranged from 3 to 9.




\subsection{Synthesis Speed}
Performing synthesis in less than a minute enables interactive use by programmers, and even faster times would support runtime compilation. We measured the time to synthesize several attribute grammars. Figure~\ref{fig:synthperf} shows the lines of code for each one and various timings on a 2.66GHz Intel Core i7 with 4GB of RAM.

Generally, synthesizing a schedule, whether an arbitrary one (\texttt{1st}) or from a sketch specifying the traversal sequence (\texttt{sketch}), takes less than 30 seconds. The exception was CSS, which we discuss later and was still fast. 

Emitting all schedules is even faster per emitted schedule (\texttt{avg}) than just finding the first. While the total time to emit all schedules can be slow, we note that enumeration is for offline autotuning. Finally, the greedy heuristic was necessary for enumerating schedules. Even after one day of running the non-greedy algorithm for CSS, most of the greedy CSS schedules were still not emitted. 

Overall, we see that synthesis is fast enough for interactive use by the programmer.


\subsection{Autotuning}

We evaluated schedule autotuning speedups for \hlangpp{} on \texttt{laptop}:

\paragraph*{Comparing greedy schedules} We enumerated greedy schedules for \hlangpp{} and compared performance on 1 and 2 cores. The relative standard deviation for performance of different schedules ($\sigma / \mu$) is 8\%. The best schedules for 1 and 2 cores are different. Swapping them leads to 20-30\% performance degradation, and the difference between the best and worst schedules for the two scenarios are 32\% and 42\%, respectively. Autotuning schedules improves performance. 

\paragraph*{Comparing greedy to non-greedy}
Our schedule enumeration is not exhaustive because of the greedy heuristic, and therefore may miss fast schedules (Section~\ref{sec:check}).  For a fixed schedule of traversals with a greedy attribute schedule, non-greedy attribute schedules were 0-6\% faster. On average, however,  non-greedy schedules were 5\% slower. Greedy scheduling was safe for \hlangpp{}.

In our case studies, much of the benefit of autotuning derives from trying to greedy schedules: one that starts with a parallel preorder traversal and then alternatives with parallel postorder ones, and vice-versa. We did see exceptions however, such as CSS benefiting from a nested traversal, and some grammars requiring occasional repetition of preorder and postorder traversals. 


%
%
%\chapter{Interacting with Automatic Parallelizers through Schedule Sketching}
%
%\section{Automatic Parallelization: The Good, the Bad, and the Ugly}
%
%\subsection{The Good: Automating Dependency Management}
%\subsection{The Bad: Guiding Parallelization}
%\subsection{The Ugly: Preventing Serialization}
%
%\section{Holes}
%\section{Generalizing Holes to Unification}
%
%\section{Case Studies: Sketching in Action}
%Show use in CSS and data viz: 
%\begin{itemize}
%\item when automatic is fine
%\item when sketch needed for checking/debugging
%\item when sketch needed for sharing
%\end{itemize}
%
%\section{Related Work}
%\begin{itemize}
%\item sketch, sketch for concurrent structures
%\item oopsla paper for individual traversals
%\end{itemize}
