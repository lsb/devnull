% (This is included by thesis.tex; you do not latex it by itself.)

\begin{abstract}

% The text of the abstract goes here.  If you need to use a \section
% command you will need to use \section*, \subsection*, etc. so that
% you don't get any numbering.  You probably won't be using any of
% these commands in the abstract anyway.

From low-power phones to speed-hungry data visualizations, web browsers need a performance boost. Parallelization is an attractive opportunity because commodity client devices already feature multicore, subword-SIMD, and GPU hardware. However, a typical webpage will not strongly benefit from modern hardware because browsers were only designed for sequential execution. We therefore need to redesign browsers to be parallel. This thesis focuses on a browser component that we found to be particularly challenging to implement: the layout engine. 

We address layout engine implementation by identifying its surprising connection with attribute grammars and then solving key ensuing challenges:
\begin{enumerate}
\item We show how layout engines, both for documents and data visualization, can often be functionally specified in our extended form of attribute grammars. 
\item We introduce a synthesizer that automatically schedules an attribute grammar as a composition of parallel tree traversals. Notably, our synthesizer is fast, simple to extend, and finds schedules that assist aggressive code generation.
\item We make editing parallel code safe by introducing a simple programming construct for partial behavioral specification: schedule sketching. 
\item We optimize tree traversals for SIMD, MIMD, and GPU architectures at tree load time through novel optimizations for data representation and task scheduling.
\end{enumerate}
Put together, we generated a parallel CSS document layout engine that can mostly render complex sites such as Wikipedia. Furthermore, we scripted data visualizations that support interacting with over 100,000 data points in real time.

\end{abstract}
